\documentclass{article}
\usepackage{tikz}
\usetikzlibrary{shapes,arrows}

\begin{document}

% Define block styles
\tikzstyle{startstop} = [rectangle, rounded corners, minimum width=3cm, minimum height=1cm,text centered, draw=black, fill=red!30]
\tikzstyle{io} = [trapezium, trapezium left angle=70, trapezium right angle=110, minimum width=3cm, minimum height=1cm, text centered, draw=black, fill=blue!30]
\tikzstyle{process} = [rectangle, minimum width=5cm, minimum height=1cm, text centered, draw=black, fill=orange!30]
\tikzstyle{decision} = [diamond, aspect=3, minimum width=3cm, minimum height=1cm, text centered, draw=black, fill=green!30]
\tikzstyle{arrow} = [thick,->,>=stealth]

\begin{tikzpicture}[node distance=2cm]

\node (start) [startstop] {Start};
\node (input1) [io, below of=start] {Masukkan nomor inventaris};
\node (decision) [decision, below of=input1, yshift=-1.5cm] {Cek Status Alat};
\node (process1) [process, below of=decision, yshift=-1.5cm] {Alat Tersedia};
\node (process2) [process, right of=decision, xshift=4cm] {Alat tidak Tersedia};
\node (process3) [process, below of=process1] {Hitung Denda (Jika Ada)};
\node (stop) [startstop, below of=process3] {Selesai};

\draw [arrow] (start) -- (input1);
\draw [arrow] (input1) -- (decision);
\draw [arrow] (decision) -- node[anchor=west] {dipinjam} (process1);
\draw [arrow] (decision) -- node[anchor=south] {!dipinjam } (process2);
\draw [arrow] (process1) -- (process3);
\draw [arrow] (process2) |- (stop);
\draw [arrow] (process3) -- (stop);

\end{tikzpicture}

\end{document}
