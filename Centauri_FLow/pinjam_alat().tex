\documentclass{article}
\usepackage{tikz}
\usetikzlibrary{shapes,arrows}

\begin{document}

% Define block styles
\tikzstyle{startstop} = [rectangle, rounded corners, minimum width=3cm, minimum height=1cm,text centered, draw=black, fill=red!30]
\tikzstyle{io} = [trapezium, trapezium left angle=70, trapezium right angle=110, minimum width=3cm, minimum height=1cm, text centered, draw=black, fill=blue!30]
\tikzstyle{process} = [rectangle, minimum width=5cm, minimum height=1cm, text centered, draw=black, fill=orange!30]
\tikzstyle{decision} = [diamond, minimum width=3cm, minimum height=1cm, text centered, draw=black, fill=green!30]
\tikzstyle{arrow} = [thick,->,>=stealth]

\begin{tikzpicture}[node distance=2cm]

\node (start) [startstop] {Start};
\node (input1) [io, below of=start] {Masukkan nomor inventaris};
\node (decision) [decision, below of=input1, yshift=-1.5cm] {Cek Ketersediaan Alat};
\node (process) [process, below of=decision, yshift=-1.5cm] {Alat Tersedia & Dipinjam};
\node (stop) [startstop, below of=process, yshift=-0.5cm] {Selesai};

\draw [arrow] (start) -- (input1);
\draw [arrow] (input1) -- (decision);
\draw [arrow] (decision) -- node[anchor=west] {Ya} (process);
\draw [arrow] (process) -- (stop);
\draw [arrow] (decision.east) -- ++(2,0) |- node[anchor=south east] {Tidak} (stop.east);

\end{tikzpicture}

\end{document}
