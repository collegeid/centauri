\documentclass{article}
\usepackage{tikz}
\usetikzlibrary{shapes.geometric, arrows}

\begin{document}

\tikzstyle{startstop} = [rectangle, rounded corners, minimum width=3cm, minimum height=1cm,text centered, draw=black, fill=red!30]
\tikzstyle{io} = [trapezium, trapezium left angle=70, trapezium right angle=110, minimum width=3cm, minimum height=1cm, text centered, draw=black, fill=blue!30]
\tikzstyle{process} = [rectangle, minimum width=3cm, minimum height=1cm, text centered, draw=black, fill=orange!30]
\tikzstyle{decision} = [diamond, minimum width=3cm, minimum height=1cm, text centered, draw=black, fill=green!30]
\tikzstyle{arrow} = [thick,->,>=stealth]

\begin{tikzpicture}[node distance=2cm]

\node (start) [startstop] {Registrasi Anggota};
\node (input1) [io, below of=start] {Masukkan Nama Kamu};
\node (input2) [io, below of=input1] {Masukkan Alamat/Domisili};
\node (input3) [io, below of=input2] {Masukkan Nomor Hp/Wahtsapp};
\node (input4) [io, below of=input3] {Masukkan NIK (16 Digits)};
\node (input5) [io, below of=input4] {Masukkan Username Baru};
\node (input6) [io, below of=input5] {Masukkan Password Baru};
\node (process) [process, below of=input6] {Tambahkan Data ke Daftar Anggota};
\node (stop) [startstop, below of=process] {Selesai};

\draw [arrow] (start) -- (input1);
\draw [arrow] (input1) -- (input2);
\draw [arrow] (input2) -- (input3);
\draw [arrow] (input3) -- (input4);
\draw [arrow] (input4) -- (input5);
\draw [arrow] (input5) -- (input6);
\draw [arrow] (input6) -- (process);
\draw [arrow] (process) -- (stop);

\end{tikzpicture}

\end{document}
