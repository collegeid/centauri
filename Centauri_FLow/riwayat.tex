\documentclass{article}
\usepackage{tikz}
\usetikzlibrary{shapes,arrows}

\begin{document}

% Define block styles
\tikzstyle{startstop} = [rectangle, rounded corners, minimum width=3cm, minimum height=1cm,text centered, draw=black, fill=red!30]
\tikzstyle{io} = [trapezium, trapezium left angle=70, trapezium right angle=110, minimum width=3cm, minimum height=1cm, text centered, draw=black, fill=blue!30]
\tikzstyle{process} = [rectangle, minimum width=5cm, minimum height=1cm, text centered, draw=black, fill=orange!30]
\tikzstyle{decision} = [diamond, aspect=3, minimum width=3cm, minimum height=1cm, text centered, draw=black, fill=green!30]
\tikzstyle{arrow} = [thick,->,>=stealth]

\begin{tikzpicture}[node distance=2cm]

\node (start) [startstop] {Start};
\node (decision1) [decision, below of=start, yshift=-1.5cm] {Periksa Level User};
\node (decision2) [decision, below of=decision1, yshift=-2.5cm] {Pilihan Riwayat};
\node (process1) [process, below of=decision2, xshift=-5cm] {Riwayat Pribadi};
\node (process2) [process, below of=decision2, xshift=5cm] {Riwayat Keseluruhan};
\node (process3) [process, below of=process1] {Tampilkan Riwayat};
\node (process4) [process, below of=process2] {Tampilkan Riwayat};

\draw [arrow] (start) -- (decision1);
\draw [arrow] (decision1) -- (decision2);
\draw [arrow] (decision2.west) -- ++(-2.5,0) -- ++(0,2) -| node[anchor=south] {Pilihan = 1} (process1);
\draw [arrow] (decision2.east) -- ++(2.5,0) -- ++(0,2) -| node[anchor=south] {Pilihan = 2} (process2);
\draw [arrow] (decision2) -- node[anchor=east, xshift=-0.5cm] {Admin atau Staff} (process1);
\draw [arrow] (decision2) -- node[anchor=west, xshift=0.5cm] {Admin atau Staff} (process2);
\draw [arrow] (process1) -- (process3);
\draw [arrow] (process2) -- (process4);

\end{tikzpicture}

\end{document}
